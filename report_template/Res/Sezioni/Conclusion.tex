\section{Conclusion 5\%}
We proposed a solution for the detection and the classification of European traffic signs. We explored the world of autonomous vehicles, a new and growing topic. We adopted the modern state-of-the-art YOLO algorithm to build a fast object detection system, but it also isn't always very accurate like other CNN based system which focuses a lot on the objects to detect but at the same time are much slower. This work is very complex due to the different factors that can influence the detection or the classification.

The strengths of our system are extensibility and modularity. We introduced a 2 step detection and classification model, this mean that the detection task and the classification task are independent from each other. One could change a model without modifying the other or changing the system's architecture. It is also possible to easily extend the number of classes of the classifier.

The drafting of this paper helped us to understand and to exercise on finding new solutions and carrying out experiments quickly, establishing metrics for their evaluation and comparison. The operation that took the longest time was the training of models and that made more difficult trying out new ideas or experiments. We learned the state-of-the-art YOLO algorithm for object detection, making use of methods from the libraries OpenCV and Keras. We also experienced how to deal with papers extracting valuable information and applying it in our work.

In future work we intent to:
\begin{itemize}
	\item Test the model using the newest YOLOv4, the recent release of the fourth version of YOLO algorithm which should offers better performance;
	\item Expand the dataset including new traffic sign classes or expanding classes already present like speed limit signals in order to build a more precise model;
	\item Add images for that classes which suffer the problem of intra-class appearance differences;
	\item Use custom loss function to obtain more accurate detections;
	\item Try different data augmentation settings to improve the classifier accuracy, for example introducing photo-metric distortions like changing the brightness, saturation, contrast and noise or trying other interesting techniques of augmenting the images like CutOut\cite{cutout} which randomly masks out square regions of input during training  or Random Erasing\cite{random} which selects rectangle regions in an image and erases its pixels with random values, in order to improve robustness and performance of CNNs;
	\item Try to implement the classifier through a functional model using the Keras functional API \footnote{\url{https://keras.io/guides/functional_api}}  which is a way to create models that are more flexible than the Sequential API. In fact, can handle models with non-linear topology, shared layers, and even multiple inputs or outputs in order to build graphs of layers.
\end{itemize}



