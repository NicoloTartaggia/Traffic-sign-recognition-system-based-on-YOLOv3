\section{Introduction 10\%}
Describe the problem you are working on, why it's important, and an overview of your results.\\
Image processing and its related algorithms are used to process digital images. Although it may seem recent technology, many of the techniques were developed in the 1960s. However, computers were expensive at this time and the digital imaging process required too much resources for many to even consider. The situation changed in the 1970s, when digital image processing began to be used massively as cheaper computers and dedicated hardware became available. With the accessibility to faster computers in the 2000s, digital image processing has become the most common form of image processing, able to compute different tasks such as image classification, video and real time detection. With the birth and growth of intelligent systems, Computer vision is increasingly used in the field of intelligent transport, which includes the subject of this paper: traffic sign recognition. These systems are typically based on detecting a region of interest (ROI), in which the traffic sign is located, recognising typical characteristics such as colour and geometric form.
Traffic sign recognition can be seen as a third eye which concerns a wide area of usages such as self-driving cars.\\

The application developed includes not only the standard functionality of image processing but also image and video detection and image classification. The original idea of combining different processes shows how powerful this tool can be.\\

The system described in this paper detects the area of the original image where the potential traffic sign is located, and tries to classify it. This is made possible through a long and accurate training process, based on a dataset of normalized traffic signs. TODO: The process starts with a preprocessing stage for the input image, where the parameters of the image, such as resolution or contrast, are modified to guarantee that the filters and algorithms used later behave properly. When the parameters of the image have been adjusted, an edge detection algorithm is used in order to determine the potential areas of the image where a possible traffic sign may be located. The next step is to separate the regions of interest (ROIs) found in the image and obtain the potential traffic signs. Every potential traffic sign is submitted to a recognition process, using a cross-correlation algorithm that compares each one with a database, which contains patterns of traffic signs. This software includes a graphical user interface, which allows the user to control each stage of the application. The results obtained show a high success rate, dependent on the environmental conditions of the input image and its resolution
%------------------------------------------------------------------------